\documentclass[]{article}
\usepackage{lmodern}
\usepackage{amssymb,amsmath}
\usepackage{ifxetex,ifluatex}
\usepackage{fixltx2e} % provides \textsubscript
\ifnum 0\ifxetex 1\fi\ifluatex 1\fi=0 % if pdftex
  \usepackage[T1]{fontenc}
  \usepackage[utf8]{inputenc}
\else % if luatex or xelatex
  \ifxetex
    \usepackage{mathspec}
    \usepackage{xltxtra,xunicode}
  \else
    \usepackage{fontspec}
  \fi
  \defaultfontfeatures{Mapping=tex-text,Scale=MatchLowercase}
  \newcommand{\euro}{€}
\fi
% use upquote if available, for straight quotes in verbatim environments
\IfFileExists{upquote.sty}{\usepackage{upquote}}{}
% use microtype if available
\IfFileExists{microtype.sty}{%
\usepackage{microtype}
\UseMicrotypeSet[protrusion]{basicmath} % disable protrusion for tt fonts
}{}
\ifxetex
  \usepackage[setpagesize=false, % page size defined by xetex
              unicode=false, % unicode breaks when used with xetex
              xetex]{hyperref}
\else
  \usepackage[unicode=true]{hyperref}
\fi
\hypersetup{breaklinks=true,
            bookmarks=true,
            pdfauthor={},
            pdftitle={},
            colorlinks=true,
            citecolor=blue,
            urlcolor=blue,
            linkcolor=magenta,
            pdfborder={0 0 0}}
\urlstyle{same}  % don't use monospace font for urls
\setlength{\parindent}{0pt}
\setlength{\parskip}{6pt plus 2pt minus 1pt}
\setlength{\emergencystretch}{3em}  % prevent overfull lines
\setcounter{secnumdepth}{0}

\date{}

\begin{document}

\section{psilo}\label{psilo}

a parallel, safe, inferencing list operation language for writing
interesting programs. \href{https://github.com/gatlin/psilo}{View it on
GitHub.}

© 2014 \href{http://niltag.net}{Gatlin Johnson}
\href{mailto:gatlin@niltag.net}{\nolinkurl{gatlin@niltag.net}}

\section{What is psilo?}\label{what-is-psilo}

psilo will be a programming language created with the philosophy that
\emph{all} programs essentially define (restricted) languages.

Technical Features (planned):

\begin{itemize}
\itemsep1pt\parskip0pt\parsep0pt
\item
  No run-time garbage collection necessary owing to uniqueness types
\item
  Static typing for compile-time verification and optimization
\item
  Malleable syntax with macros
\item
  Dead-simple parallelism with special array types
\item
  Monadic continuations and iteratee composition made dead simple
\item
  Orthogonal core syntax and semantics for your performance and my
  sanity
\end{itemize}

Philosophy:

\begin{itemize}
\itemsep1pt\parskip0pt\parsep0pt
\item
  All programming is manipulating languages.
\item
  Types define grammars; functions define parsers.
\item
  The earlier a question may be answered, the better.
\item
  If the computer can do it, it should.
\end{itemize}

\section{Status}\label{status}

Psilo is still being designed. I have implemented a really simple
evaluator as well as this website so that when the time comes, I won't
have tedious tooling or process issues getting in the way of
implementation.

Additionally, the simple interpreter might be of educational value.

\section{How to build}\label{how-to-build}

You need the Glasgow Haskell Compiler and a number of libraries; I
suggest starting off with \href{http://haskell.org/platform}{the Haskell
platform}.

Clone the repository:

\begin{verbatim}
git clone https://github.com/gatlin/psilo
\end{verbatim}

Set up a cabal sandbox:

\begin{verbatim}
cabal sandbox init
cabal configure
cabal install --only-dependencies
\end{verbatim}

Then make with:

\begin{verbatim}
make
\end{verbatim}

And return to the Edenic, pre-build post-checkout status of the code
with

\begin{verbatim}
make clean
\end{verbatim}

\section{Questions / comments / hate
mail}\label{questions-comments-hate-mail}

Use the Issues feature of GitHub or email me:
\href{mailto:gatlin@niltag.net}{\nolinkurl{gatlin@niltag.net}}.

\end{document}
