\documentclass[]{article}
\usepackage{lmodern}
\usepackage{amssymb,amsmath}
\usepackage{ifxetex,ifluatex}
\usepackage{fixltx2e} % provides \textsubscript
\ifnum 0\ifxetex 1\fi\ifluatex 1\fi=0 % if pdftex
  \usepackage[T1]{fontenc}
  \usepackage[utf8]{inputenc}
\else % if luatex or xelatex
  \ifxetex
    \usepackage{mathspec}
    \usepackage{xltxtra,xunicode}
  \else
    \usepackage{fontspec}
  \fi
  \defaultfontfeatures{Mapping=tex-text,Scale=MatchLowercase}
  \newcommand{\euro}{€}
\fi
% use upquote if available, for straight quotes in verbatim environments
\IfFileExists{upquote.sty}{\usepackage{upquote}}{}
% use microtype if available
\IfFileExists{microtype.sty}{%
\usepackage{microtype}
\UseMicrotypeSet[protrusion]{basicmath} % disable protrusion for tt fonts
}{}
\usepackage{color}
\usepackage{fancyvrb}
\newcommand{\VerbBar}{|}
\newcommand{\VERB}{\Verb[commandchars=\\\{\}]}
\DefineVerbatimEnvironment{Highlighting}{Verbatim}{commandchars=\\\{\}}
% Add ',fontsize=\small' for more characters per line
\newenvironment{Shaded}{}{}
\newcommand{\KeywordTok}[1]{\textcolor[rgb]{0.00,0.44,0.13}{\textbf{{#1}}}}
\newcommand{\DataTypeTok}[1]{\textcolor[rgb]{0.56,0.13,0.00}{{#1}}}
\newcommand{\DecValTok}[1]{\textcolor[rgb]{0.25,0.63,0.44}{{#1}}}
\newcommand{\BaseNTok}[1]{\textcolor[rgb]{0.25,0.63,0.44}{{#1}}}
\newcommand{\FloatTok}[1]{\textcolor[rgb]{0.25,0.63,0.44}{{#1}}}
\newcommand{\CharTok}[1]{\textcolor[rgb]{0.25,0.44,0.63}{{#1}}}
\newcommand{\StringTok}[1]{\textcolor[rgb]{0.25,0.44,0.63}{{#1}}}
\newcommand{\CommentTok}[1]{\textcolor[rgb]{0.38,0.63,0.69}{\textit{{#1}}}}
\newcommand{\OtherTok}[1]{\textcolor[rgb]{0.00,0.44,0.13}{{#1}}}
\newcommand{\AlertTok}[1]{\textcolor[rgb]{1.00,0.00,0.00}{\textbf{{#1}}}}
\newcommand{\FunctionTok}[1]{\textcolor[rgb]{0.02,0.16,0.49}{{#1}}}
\newcommand{\RegionMarkerTok}[1]{{#1}}
\newcommand{\ErrorTok}[1]{\textcolor[rgb]{1.00,0.00,0.00}{\textbf{{#1}}}}
\newcommand{\NormalTok}[1]{{#1}}
\ifxetex
  \usepackage[setpagesize=false, % page size defined by xetex
              unicode=false, % unicode breaks when used with xetex
              xetex]{hyperref}
\else
  \usepackage[unicode=true]{hyperref}
\fi
\hypersetup{breaklinks=true,
            bookmarks=true,
            pdfauthor={},
            pdftitle={psilo},
            colorlinks=true,
            citecolor=blue,
            urlcolor=blue,
            linkcolor=magenta,
            pdfborder={0 0 0}}
\urlstyle{same}  % don't use monospace font for urls
\setlength{\parindent}{0pt}
\setlength{\parskip}{6pt plus 2pt minus 1pt}
\setlength{\emergencystretch}{3em}  % prevent overfull lines
\setcounter{secnumdepth}{0}

\title{psilo}
\date{}

\begin{document}
\maketitle

This is the main program outline. If an argument is present on the
command line then we execute the program in that file and halt.
Otherwise we fire up a repl.

\begin{Shaded}
\begin{Highlighting}[]
\KeywordTok{module} \DataTypeTok{Main} \KeywordTok{where}

\KeywordTok{import }\DataTypeTok{Parser}
\KeywordTok{import }\DataTypeTok{Syntax}
\KeywordTok{import }\DataTypeTok{Evaluator}

\KeywordTok{import }\DataTypeTok{Control.Monad.Trans}
\KeywordTok{import }\DataTypeTok{System.Console.Haskeline}

\KeywordTok{import }\DataTypeTok{System.Environment}
\KeywordTok{import }\DataTypeTok{System.IO}
\end{Highlighting}
\end{Shaded}

\texttt{eval} amounts to taking a line of code (a line in the repl, an
expression otherwise), getting the \texttt{Expr} value from the parser
and then running a \texttt{Machine} with said value.

The result is the state of the machine after it has been run.

\begin{Shaded}
\begin{Highlighting}[]
\OtherTok{eval ::} \DataTypeTok{String} \OtherTok{->} \DataTypeTok{MStore} \OtherTok{->} \DataTypeTok{IO} \DataTypeTok{MStore}
\NormalTok{eval line store }\FunctionTok{=} \KeywordTok{do}
    \KeywordTok{let} \NormalTok{res }\FunctionTok{=} \NormalTok{parseTopLevel line}
    \KeywordTok{case} \NormalTok{res }\KeywordTok{of}
        \DataTypeTok{Left} \NormalTok{err }\OtherTok{->} \NormalTok{print err }\FunctionTok{>>} \NormalTok{return store}
        \DataTypeTok{Right} \NormalTok{[ex] }\OtherTok{->} \NormalTok{execute (}\OtherTok{ex ::} \DataTypeTok{Expr} \NormalTok{()) }\FunctionTok{>>=} \NormalTok{return}

    \KeywordTok{where} \NormalTok{execute v }\FunctionTok{=} \KeywordTok{do}
              \NormalTok{(val, store') }\OtherTok{<-} \NormalTok{runMachine }\FunctionTok{.} \NormalTok{interpret }\FunctionTok{$} \NormalTok{v}
              \NormalTok{putStrLn }\FunctionTok{.} \NormalTok{show }\FunctionTok{$} \NormalTok{val}
              \NormalTok{return store'}
\end{Highlighting}
\end{Shaded}

The repl is nothing more than calling \texttt{eval} in an endless loop.

\begin{Shaded}
\begin{Highlighting}[]
\OtherTok{repl ::} \DataTypeTok{IO} \NormalTok{()}
\NormalTok{repl }\FunctionTok{=} \NormalTok{runInputT defaultSettings (loop initialStore) }\KeywordTok{where}
    \NormalTok{loop store }\FunctionTok{=} \KeywordTok{do}
        \NormalTok{minput }\OtherTok{<-} \NormalTok{getInputLine }\StringTok{"psilo> "}
        \KeywordTok{case} \NormalTok{minput }\KeywordTok{of}
            \DataTypeTok{Nothing} \OtherTok{->} \NormalTok{outputStrLn }\StringTok{"Goodbye."}
            \DataTypeTok{Just} \NormalTok{input }\OtherTok{->} \KeywordTok{do}
                \KeywordTok{case} \NormalTok{input }\KeywordTok{of}
                    \StringTok{":state"} \OtherTok{->} \NormalTok{liftIO (putStrLn }\FunctionTok{.} \NormalTok{show }\FunctionTok{$} \NormalTok{store) }\FunctionTok{>>} \NormalTok{loop store}
                    \NormalTok{_        }\OtherTok{->} \KeywordTok{do}
                        \NormalTok{store' }\OtherTok{<-} \NormalTok{liftIO }\FunctionTok{$} \NormalTok{eval input store}
                        \NormalTok{loop store'}
\end{Highlighting}
\end{Shaded}

If we execute \texttt{eval} on the contents of a file of code, we have a
normal interpreter:

\begin{Shaded}
\begin{Highlighting}[]
\OtherTok{execFile ::} \DataTypeTok{String} \OtherTok{->} \DataTypeTok{IO} \NormalTok{()}
\NormalTok{execFile fname }\FunctionTok{=} \NormalTok{readFile fname }\FunctionTok{>>=} \NormalTok{(flip eval) initialStore }\FunctionTok{>>} \NormalTok{return ()}
\end{Highlighting}
\end{Shaded}

\begin{Shaded}
\begin{Highlighting}[]
\OtherTok{main ::} \DataTypeTok{IO} \NormalTok{()}
\NormalTok{main }\FunctionTok{=} \KeywordTok{do}
    \NormalTok{args }\OtherTok{<-} \NormalTok{getArgs}
    \KeywordTok{case} \NormalTok{args }\KeywordTok{of}
        \NormalTok{[]      }\OtherTok{->} \NormalTok{repl }\FunctionTok{>>} \NormalTok{return ()}
        \NormalTok{[fname] }\OtherTok{->} \NormalTok{execFile fname }\FunctionTok{>>} \NormalTok{return ()}
\end{Highlighting}
\end{Shaded}

\end{document}
