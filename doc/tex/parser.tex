\documentclass[]{article}
\usepackage{lmodern}
\usepackage{amssymb,amsmath}
\usepackage{ifxetex,ifluatex}
\usepackage{fixltx2e} % provides \textsubscript
\ifnum 0\ifxetex 1\fi\ifluatex 1\fi=0 % if pdftex
  \usepackage[T1]{fontenc}
  \usepackage[utf8]{inputenc}
\else % if luatex or xelatex
  \ifxetex
    \usepackage{mathspec}
    \usepackage{xltxtra,xunicode}
  \else
    \usepackage{fontspec}
  \fi
  \defaultfontfeatures{Mapping=tex-text,Scale=MatchLowercase}
  \newcommand{\euro}{€}
\fi
% use upquote if available, for straight quotes in verbatim environments
\IfFileExists{upquote.sty}{\usepackage{upquote}}{}
% use microtype if available
\IfFileExists{microtype.sty}{%
\usepackage{microtype}
\UseMicrotypeSet[protrusion]{basicmath} % disable protrusion for tt fonts
}{}
\usepackage{color}
\usepackage{fancyvrb}
\newcommand{\VerbBar}{|}
\newcommand{\VERB}{\Verb[commandchars=\\\{\}]}
\DefineVerbatimEnvironment{Highlighting}{Verbatim}{commandchars=\\\{\}}
% Add ',fontsize=\small' for more characters per line
\newenvironment{Shaded}{}{}
\newcommand{\KeywordTok}[1]{\textcolor[rgb]{0.00,0.44,0.13}{\textbf{{#1}}}}
\newcommand{\DataTypeTok}[1]{\textcolor[rgb]{0.56,0.13,0.00}{{#1}}}
\newcommand{\DecValTok}[1]{\textcolor[rgb]{0.25,0.63,0.44}{{#1}}}
\newcommand{\BaseNTok}[1]{\textcolor[rgb]{0.25,0.63,0.44}{{#1}}}
\newcommand{\FloatTok}[1]{\textcolor[rgb]{0.25,0.63,0.44}{{#1}}}
\newcommand{\CharTok}[1]{\textcolor[rgb]{0.25,0.44,0.63}{{#1}}}
\newcommand{\StringTok}[1]{\textcolor[rgb]{0.25,0.44,0.63}{{#1}}}
\newcommand{\CommentTok}[1]{\textcolor[rgb]{0.38,0.63,0.69}{\textit{{#1}}}}
\newcommand{\OtherTok}[1]{\textcolor[rgb]{0.00,0.44,0.13}{{#1}}}
\newcommand{\AlertTok}[1]{\textcolor[rgb]{1.00,0.00,0.00}{\textbf{{#1}}}}
\newcommand{\FunctionTok}[1]{\textcolor[rgb]{0.02,0.16,0.49}{{#1}}}
\newcommand{\RegionMarkerTok}[1]{{#1}}
\newcommand{\ErrorTok}[1]{\textcolor[rgb]{1.00,0.00,0.00}{\textbf{{#1}}}}
\newcommand{\NormalTok}[1]{{#1}}
\ifxetex
  \usepackage[setpagesize=false, % page size defined by xetex
              unicode=false, % unicode breaks when used with xetex
              xetex]{hyperref}
\else
  \usepackage[unicode=true]{hyperref}
\fi
\hypersetup{breaklinks=true,
            bookmarks=true,
            pdfauthor={},
            pdftitle={The Parser},
            colorlinks=true,
            citecolor=blue,
            urlcolor=blue,
            linkcolor=magenta,
            pdfborder={0 0 0}}
\urlstyle{same}  % don't use monospace font for urls
\setlength{\parindent}{0pt}
\setlength{\parskip}{6pt plus 2pt minus 1pt}
\setlength{\emergencystretch}{3em}  % prevent overfull lines
\setcounter{secnumdepth}{0}

\title{The Parser}
\date{}

\begin{document}
\maketitle

This implements the basic s-expression syntax along with some sugar,
like \texttt{let} bindings.

The result of one of the top-level parsing functions is a
\texttt{Parser (Expr a)} value from which the \texttt{Expr} value may be
extracted and given to the evaluator.

This is probably sub-optimal; parsec is a harsh master.

\begin{Shaded}
\begin{Highlighting}[]
\KeywordTok{module} \DataTypeTok{Parser} \KeywordTok{where}

\KeywordTok{import }\DataTypeTok{Text.Parsec}
\KeywordTok{import }\DataTypeTok{Text.Parsec.String} \NormalTok{(}\DataTypeTok{Parser}\NormalTok{)}
\KeywordTok{import }\DataTypeTok{Control.Applicative} \NormalTok{((<$>))}
\KeywordTok{import }\DataTypeTok{Control.Monad} \NormalTok{(mapAndUnzipM)}
\KeywordTok{import }\DataTypeTok{Control.Monad.Free}
\KeywordTok{import }\DataTypeTok{Data.List.Split} \NormalTok{(splitOn)}

\KeywordTok{import qualified} \DataTypeTok{Text.Parsec.Expr} \KeywordTok{as} \DataTypeTok{Ex}
\KeywordTok{import qualified} \DataTypeTok{Text.Parsec.Token} \KeywordTok{as} \DataTypeTok{Tok}

\KeywordTok{import }\DataTypeTok{Syntax}
\KeywordTok{import }\DataTypeTok{Lexer}

\OtherTok{parseNumber ::} \DataTypeTok{Parser} \NormalTok{(}\DataTypeTok{Expr} \NormalTok{a)}
\NormalTok{parseNumber }\FunctionTok{=} \NormalTok{try ( }\KeywordTok{do} \NormalTok{\{ n }\OtherTok{<-} \NormalTok{integer}
                       \NormalTok{; return }\FunctionTok{$} \DataTypeTok{Free} \FunctionTok{$} \DataTypeTok{AInteger} \NormalTok{n}
                       \NormalTok{\} )}
\end{Highlighting}
\end{Shaded}

Booleans are represented by the atoms \texttt{\#t} and \texttt{\#f}.

\begin{Shaded}
\begin{Highlighting}[]
\OtherTok{parseBoolean ::} \DataTypeTok{Parser} \NormalTok{(}\DataTypeTok{Expr} \NormalTok{a)}
\NormalTok{parseBoolean }\FunctionTok{=} \KeywordTok{do}
    \NormalTok{char }\CharTok{'#'}
    \NormalTok{b }\OtherTok{<-} \NormalTok{char }\CharTok{'t'} \FunctionTok{<|>} \NormalTok{char }\CharTok{'f'}
    \KeywordTok{case} \NormalTok{b }\KeywordTok{of}
        \CharTok{'t'} \OtherTok{->} \NormalTok{return }\FunctionTok{$} \DataTypeTok{Free} \FunctionTok{$} \DataTypeTok{ABoolean} \DataTypeTok{True}
        \CharTok{'f'} \OtherTok{->} \NormalTok{return }\FunctionTok{$} \DataTypeTok{Free} \FunctionTok{$} \DataTypeTok{ABoolean} \DataTypeTok{False}
\end{Highlighting}
\end{Shaded}

Symbols are like ``atoms'' in other lisps or Erlang. They are equivalent
only to themselves and have no intrinsic value. They are mostly used to
bind values in lambda abstractions.

\begin{Shaded}
\begin{Highlighting}[]
\OtherTok{parseSymbol ::} \DataTypeTok{Parser} \NormalTok{(}\DataTypeTok{Expr} \NormalTok{a)}
\NormalTok{parseSymbol }\FunctionTok{=} \KeywordTok{do}
    \NormalTok{sym }\OtherTok{<-} \NormalTok{operator }\FunctionTok{<|>} \NormalTok{identifier}
    \NormalTok{sym' }\OtherTok{<-} \NormalTok{chomped sym}
    \NormalTok{return }\FunctionTok{$} \DataTypeTok{Free} \FunctionTok{$} \DataTypeTok{ASymbol} \NormalTok{sym'}
    \KeywordTok{where} \NormalTok{chomped s }\FunctionTok{=} \KeywordTok{let} \NormalTok{s' }\FunctionTok{=} \NormalTok{splitOn }\StringTok{":"} \NormalTok{s}
                      \KeywordTok{in}  \NormalTok{return }\FunctionTok{$} \NormalTok{s' }\FunctionTok{!!} \DecValTok{0}
\end{Highlighting}
\end{Shaded}

Lamba abstractions, or \emph{functions}. A function definition is a list
of symbols to bind to the elements of the argument list, and a psilo
expression to evaluate in the context of the arguments.

\begin{Shaded}
\begin{Highlighting}[]
\OtherTok{parseFn ::} \DataTypeTok{Parser} \NormalTok{(}\DataTypeTok{Expr} \NormalTok{a)}
\NormalTok{parseFn }\FunctionTok{=} \KeywordTok{do}
    \NormalTok{reserved }\StringTok{"\textbackslash{}\textbackslash{}"}
    \NormalTok{optional whitespace}
    \NormalTok{(}\DataTypeTok{Free} \NormalTok{(}\DataTypeTok{AList} \NormalTok{arg)) }\OtherTok{<-} \NormalTok{parens parseQuotedList}
    \NormalTok{optional whitespace}
    \NormalTok{body }\OtherTok{<-} \NormalTok{parseExpr}
    \NormalTok{return }\FunctionTok{$} \DataTypeTok{Free} \FunctionTok{$} \DataTypeTok{ALambda} \NormalTok{(expr2symlist arg) body}
\end{Highlighting}
\end{Shaded}

\texttt{let} bindings are formally equivalent to wrapping an expression
in an outer closure and immediately evaluating it, like so:

\begin{Shaded}
\begin{Highlighting}[]
\OtherTok{parseLetBinding ::} \DataTypeTok{Parser} \NormalTok{(}\DataTypeTok{Expr} \NormalTok{a)}
\NormalTok{parseLetBinding }\FunctionTok{=} \KeywordTok{do}
    \NormalTok{optional whitespace}
    \NormalTok{sym }\OtherTok{<-} \NormalTok{parseSymbol}
    \NormalTok{optional whitespace}
    \NormalTok{val }\OtherTok{<-} \NormalTok{parseExpr}
    \NormalTok{return }\FunctionTok{$} \DataTypeTok{Free} \FunctionTok{.} \DataTypeTok{AList} \FunctionTok{$} \NormalTok{[sym, val]}

\OtherTok{parseLetBindings ::} \DataTypeTok{Parser} \NormalTok{(}\DataTypeTok{Expr} \NormalTok{a)}
\NormalTok{parseLetBindings }\FunctionTok{=} \NormalTok{fmap (}\DataTypeTok{Free} \FunctionTok{.} \DataTypeTok{AList}\NormalTok{) }\FunctionTok{$} \NormalTok{parens parseLetBinding }\OtherTok{`sepBy`} \NormalTok{whitespace}

\OtherTok{parseLet ::} \DataTypeTok{Parser} \NormalTok{(}\DataTypeTok{Expr} \NormalTok{a)}
\NormalTok{parseLet }\FunctionTok{=} \KeywordTok{do}
    \NormalTok{reserved }\StringTok{"let"}
    \NormalTok{optional whitespace}
    \NormalTok{(}\DataTypeTok{Free} \NormalTok{(}\DataTypeTok{AList} \NormalTok{assns)) }\OtherTok{<-} \NormalTok{parens parseLetBindings}
    \NormalTok{body  }\OtherTok{<-} \NormalTok{parseExpr }\FunctionTok{<|>} \NormalTok{return (}\DataTypeTok{Free} \NormalTok{(}\DataTypeTok{AList} \NormalTok{[]))}
    \NormalTok{(args,operands) }\OtherTok{<-} \NormalTok{(flip mapAndUnzipM) assns }\FunctionTok{$} \NormalTok{\textbackslash{}(}\DataTypeTok{Free} \NormalTok{(}\DataTypeTok{AList} \NormalTok{(x}\FunctionTok{:}\NormalTok{y}\FunctionTok{:}\NormalTok{_))) }\OtherTok{->} \NormalTok{return (x,y)}
    \NormalTok{operands' }\OtherTok{<-} \NormalTok{return }\FunctionTok{$} \DataTypeTok{Free} \FunctionTok{$} \DataTypeTok{AList} \NormalTok{operands}
    \NormalTok{fun }\OtherTok{<-} \NormalTok{return }\FunctionTok{$} \DataTypeTok{Free} \FunctionTok{$} \DataTypeTok{ALambda} \NormalTok{(expr2symlist args) body}
    \NormalTok{return }\FunctionTok{$} \DataTypeTok{Free} \NormalTok{(}\DataTypeTok{AApply} \NormalTok{fun operands')}
\end{Highlighting}
\end{Shaded}

The application of a function to a list of arguments, a single symbol,
or an arbitrary expression value.

\begin{Shaded}
\begin{Highlighting}[]
\OtherTok{parseApp ::} \DataTypeTok{Parser} \NormalTok{(}\DataTypeTok{Expr} \NormalTok{a)}
\NormalTok{parseApp }\FunctionTok{=} \KeywordTok{do}
    \NormalTok{try (reserved }\StringTok{"apply"}\NormalTok{) }\FunctionTok{>>} \NormalTok{(}\KeywordTok{do}
        \NormalTok{optional whitespace}
        \NormalTok{fun }\OtherTok{<-} \NormalTok{parseExpr}
        \NormalTok{optional whitespace}
        \NormalTok{body }\OtherTok{<-} \NormalTok{(try (char }\CharTok{'\textbackslash{}''}\NormalTok{) }\FunctionTok{>>} \NormalTok{parens parseQuotedList)}
            \FunctionTok{<|>} \NormalTok{(try (char }\CharTok{'`'}\NormalTok{) }\FunctionTok{>>} \NormalTok{parens parseUnquotable)}
        \NormalTok{return }\FunctionTok{$} \DataTypeTok{Free} \NormalTok{(}\DataTypeTok{AApply} \NormalTok{fun body))}
    \FunctionTok{<|>} \NormalTok{(}\KeywordTok{do}
        \NormalTok{fst }\OtherTok{<-} \NormalTok{try parseSymbol }\FunctionTok{<|>} \NormalTok{try (parens parseFn) }\FunctionTok{<|>} \NormalTok{parens parseApp}
        \NormalTok{optional whitespace}
        \NormalTok{rst }\OtherTok{<-} \NormalTok{fmap (}\DataTypeTok{Free} \FunctionTok{.} \DataTypeTok{AList}\NormalTok{) }\FunctionTok{$} \NormalTok{parseExpr }\OtherTok{`sepBy`} \NormalTok{whitespace}
        \NormalTok{return }\FunctionTok{$} \DataTypeTok{Free} \NormalTok{(}\DataTypeTok{AApply} \NormalTok{fst rst))}
\end{Highlighting}
\end{Shaded}

Regular list created by the quote (') operator. Enters a state where
everything is treated as a literal - no applications allowed.

\begin{Shaded}
\begin{Highlighting}[]
\OtherTok{parseQuotedList ::} \DataTypeTok{Parser} \NormalTok{(}\DataTypeTok{Expr} \NormalTok{a)}
\NormalTok{parseQuotedList }\FunctionTok{=} \NormalTok{fmap (}\DataTypeTok{Free} \FunctionTok{.} \DataTypeTok{AList}\NormalTok{) }\FunctionTok{$} \NormalTok{parseExprInQuote }\OtherTok{`sepBy`} \NormalTok{whitespace}
\end{Highlighting}
\end{Shaded}

Similar to a quoted list, except a comma operator (,) may be used to go
back into a state where application is allowed.

\begin{Shaded}
\begin{Highlighting}[]
\OtherTok{parseUnquotable ::} \DataTypeTok{Parser} \NormalTok{(}\DataTypeTok{Expr} \NormalTok{a)}
\NormalTok{parseUnquotable }\FunctionTok{=} \NormalTok{fmap (}\DataTypeTok{Free} \FunctionTok{.} \DataTypeTok{AList}\NormalTok{) }\FunctionTok{$} \NormalTok{parseExprInQuasi }\OtherTok{`sepBy`} \NormalTok{whitespace}
\end{Highlighting}
\end{Shaded}

Many things may be quoted, not just lists.

\begin{Shaded}
\begin{Highlighting}[]
\OtherTok{parseQuote ::} \DataTypeTok{Parser} \NormalTok{(}\DataTypeTok{Expr} \NormalTok{a)}
\NormalTok{parseQuote }\FunctionTok{=} \KeywordTok{do}
    \NormalTok{x }\OtherTok{<-} \NormalTok{parseSymbol }\FunctionTok{<|>} \NormalTok{parseNumber }\FunctionTok{<|>} \NormalTok{parens parseQuotedList}
    \NormalTok{return }\FunctionTok{$} \NormalTok{(}\DataTypeTok{Free} \FunctionTok{.} \DataTypeTok{AList}\NormalTok{) [(}\DataTypeTok{Free} \FunctionTok{.} \DataTypeTok{ASymbol}\NormalTok{) }\StringTok{"quote"}\NormalTok{, x]}
\end{Highlighting}
\end{Shaded}

You can quasi-quote anything you can quote, though this is of dubious
utility.

\begin{Shaded}
\begin{Highlighting}[]
\OtherTok{parseQuasi ::} \DataTypeTok{Parser} \NormalTok{(}\DataTypeTok{Expr} \NormalTok{a)}
\NormalTok{parseQuasi }\FunctionTok{=} \KeywordTok{do}
    \NormalTok{x }\OtherTok{<-} \NormalTok{parseSymbol }\FunctionTok{<|>} \NormalTok{parseNumber }\FunctionTok{<|>} \NormalTok{parens parseUnquotable}
    \NormalTok{return }\FunctionTok{$} \NormalTok{(}\DataTypeTok{Free} \FunctionTok{.} \DataTypeTok{AList}\NormalTok{) [(}\DataTypeTok{Free} \FunctionTok{.} \DataTypeTok{ASymbol}\NormalTok{) }\StringTok{"quasi"}\NormalTok{, x]}
\end{Highlighting}
\end{Shaded}

\begin{Shaded}
\begin{Highlighting}[]
\OtherTok{parseComma ::} \DataTypeTok{Parser} \NormalTok{(}\DataTypeTok{Expr} \NormalTok{a)}
\NormalTok{parseComma }\FunctionTok{=} \KeywordTok{do}
    \NormalTok{x }\OtherTok{<-} \NormalTok{parseSymbol }\FunctionTok{<|>} \NormalTok{parseExpr}
    \NormalTok{return }\FunctionTok{$} \NormalTok{(}\DataTypeTok{Free} \FunctionTok{.} \DataTypeTok{AList}\NormalTok{) [(}\DataTypeTok{Free} \FunctionTok{.} \DataTypeTok{ASymbol}\NormalTok{) }\StringTok{"comma"}\NormalTok{, x]}
\end{Highlighting}
\end{Shaded}

Top level expression parser

\begin{Shaded}
\begin{Highlighting}[]
\OtherTok{parseExpr ::} \DataTypeTok{Parser} \NormalTok{(}\DataTypeTok{Expr} \NormalTok{a)}
\NormalTok{parseExpr }\FunctionTok{=} \NormalTok{parseBoolean}
        \FunctionTok{<|>} \NormalTok{parseSymbol}
        \FunctionTok{<|>} \NormalTok{parseNumber}
        \FunctionTok{<|>} \NormalTok{(try (char }\CharTok{'\textbackslash{}''}\NormalTok{) }\FunctionTok{>>} \NormalTok{parseQuote)}
        \FunctionTok{<|>} \NormalTok{(try (char }\CharTok{'`'}\NormalTok{)  }\FunctionTok{>>} \NormalTok{parseQuasi)}
        \FunctionTok{<|>} \NormalTok{parens ( parseFn }\FunctionTok{<|>} \NormalTok{parseLet }\FunctionTok{<|>} \NormalTok{parseApp )}
\end{Highlighting}
\end{Shaded}

Expression parser inside a quoted list

\begin{Shaded}
\begin{Highlighting}[]
\OtherTok{parseExprInQuote ::} \DataTypeTok{Parser} \NormalTok{(}\DataTypeTok{Expr} \NormalTok{a)}
\NormalTok{parseExprInQuote }\FunctionTok{=} \NormalTok{parseBoolean}
               \FunctionTok{<|>} \NormalTok{parseSymbol}
               \FunctionTok{<|>} \NormalTok{parseNumber}
               \FunctionTok{<|>} \NormalTok{(try (char }\CharTok{'\textbackslash{}''}\NormalTok{) }\FunctionTok{>>} \NormalTok{parseQuote)}
               \FunctionTok{<|>} \NormalTok{parens ( parseQuotedList )}
\end{Highlighting}
\end{Shaded}

Expression parser inside a quasiquoted list

\begin{Shaded}
\begin{Highlighting}[]
\OtherTok{parseExprInQuasi ::} \DataTypeTok{Parser} \NormalTok{(}\DataTypeTok{Expr} \NormalTok{a)}
\NormalTok{parseExprInQuasi }\FunctionTok{=} \NormalTok{parseBoolean}
               \FunctionTok{<|>} \NormalTok{parseSymbol}
               \FunctionTok{<|>} \NormalTok{parseNumber}
               \FunctionTok{<|>} \NormalTok{(try (reserved }\StringTok{"'"}\NormalTok{) }\FunctionTok{>>}  \NormalTok{parseQuote)}
               \FunctionTok{<|>} \NormalTok{(try (reserved }\StringTok{"`"} \NormalTok{) }\FunctionTok{>>} \NormalTok{parseQuasi)}
               \FunctionTok{<|>} \NormalTok{(try (char }\CharTok{','}\NormalTok{)  }\FunctionTok{>>}     \NormalTok{parseComma)}
               \FunctionTok{<|>} \NormalTok{parens ( parseUnquotable )}
\end{Highlighting}
\end{Shaded}

\begin{Shaded}
\begin{Highlighting}[]
\OtherTok{contents ::} \DataTypeTok{Parser} \NormalTok{a }\OtherTok{->} \DataTypeTok{Parser} \NormalTok{a}
\NormalTok{contents p }\FunctionTok{=} \KeywordTok{do}
    \NormalTok{whitespace}
    \NormalTok{r }\OtherTok{<-} \NormalTok{p}
    \NormalTok{eof}
    \NormalTok{return r}
\end{Highlighting}
\end{Shaded}

\begin{Shaded}
\begin{Highlighting}[]
\OtherTok{topLevel ::} \DataTypeTok{Parser} \NormalTok{[}\DataTypeTok{Expr} \NormalTok{a]}
\NormalTok{topLevel }\FunctionTok{=} \NormalTok{many }\FunctionTok{$} \KeywordTok{do}
    \NormalTok{x }\OtherTok{<-} \NormalTok{parseExpr}
    \NormalTok{return x}
\end{Highlighting}
\end{Shaded}

\begin{Shaded}
\begin{Highlighting}[]
\OtherTok{doParse ::} \DataTypeTok{String} \OtherTok{->} \DataTypeTok{Either} \DataTypeTok{ParseError} \NormalTok{(}\DataTypeTok{Expr} \NormalTok{a)}
\NormalTok{doParse s }\FunctionTok{=} \NormalTok{parse (contents parseExpr) }\StringTok{"<stdin>"} \NormalTok{s}
\end{Highlighting}
\end{Shaded}

\begin{Shaded}
\begin{Highlighting}[]
\OtherTok{parseTopLevel ::} \DataTypeTok{String} \OtherTok{->} \DataTypeTok{Either} \DataTypeTok{ParseError} \NormalTok{[}\DataTypeTok{Expr} \NormalTok{a]}
\NormalTok{parseTopLevel s }\FunctionTok{=} \NormalTok{parse (contents topLevel) }\StringTok{"<stdin>"} \NormalTok{s}
\end{Highlighting}
\end{Shaded}

\end{document}
