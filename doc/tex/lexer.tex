\documentclass[]{article}
\usepackage{lmodern}
\usepackage{amssymb,amsmath}
\usepackage{ifxetex,ifluatex}
\usepackage{fixltx2e} % provides \textsubscript
\ifnum 0\ifxetex 1\fi\ifluatex 1\fi=0 % if pdftex
  \usepackage[T1]{fontenc}
  \usepackage[utf8]{inputenc}
\else % if luatex or xelatex
  \ifxetex
    \usepackage{mathspec}
    \usepackage{xltxtra,xunicode}
  \else
    \usepackage{fontspec}
  \fi
  \defaultfontfeatures{Mapping=tex-text,Scale=MatchLowercase}
  \newcommand{\euro}{€}
\fi
% use upquote if available, for straight quotes in verbatim environments
\IfFileExists{upquote.sty}{\usepackage{upquote}}{}
% use microtype if available
\IfFileExists{microtype.sty}{%
\usepackage{microtype}
\UseMicrotypeSet[protrusion]{basicmath} % disable protrusion for tt fonts
}{}
\usepackage{color}
\usepackage{fancyvrb}
\newcommand{\VerbBar}{|}
\newcommand{\VERB}{\Verb[commandchars=\\\{\}]}
\DefineVerbatimEnvironment{Highlighting}{Verbatim}{commandchars=\\\{\}}
% Add ',fontsize=\small' for more characters per line
\newenvironment{Shaded}{}{}
\newcommand{\KeywordTok}[1]{\textcolor[rgb]{0.00,0.44,0.13}{\textbf{{#1}}}}
\newcommand{\DataTypeTok}[1]{\textcolor[rgb]{0.56,0.13,0.00}{{#1}}}
\newcommand{\DecValTok}[1]{\textcolor[rgb]{0.25,0.63,0.44}{{#1}}}
\newcommand{\BaseNTok}[1]{\textcolor[rgb]{0.25,0.63,0.44}{{#1}}}
\newcommand{\FloatTok}[1]{\textcolor[rgb]{0.25,0.63,0.44}{{#1}}}
\newcommand{\CharTok}[1]{\textcolor[rgb]{0.25,0.44,0.63}{{#1}}}
\newcommand{\StringTok}[1]{\textcolor[rgb]{0.25,0.44,0.63}{{#1}}}
\newcommand{\CommentTok}[1]{\textcolor[rgb]{0.38,0.63,0.69}{\textit{{#1}}}}
\newcommand{\OtherTok}[1]{\textcolor[rgb]{0.00,0.44,0.13}{{#1}}}
\newcommand{\AlertTok}[1]{\textcolor[rgb]{1.00,0.00,0.00}{\textbf{{#1}}}}
\newcommand{\FunctionTok}[1]{\textcolor[rgb]{0.02,0.16,0.49}{{#1}}}
\newcommand{\RegionMarkerTok}[1]{{#1}}
\newcommand{\ErrorTok}[1]{\textcolor[rgb]{1.00,0.00,0.00}{\textbf{{#1}}}}
\newcommand{\NormalTok}[1]{{#1}}
\ifxetex
  \usepackage[setpagesize=false, % page size defined by xetex
              unicode=false, % unicode breaks when used with xetex
              xetex]{hyperref}
\else
  \usepackage[unicode=true]{hyperref}
\fi
\hypersetup{breaklinks=true,
            bookmarks=true,
            pdfauthor={},
            pdftitle={The Lexer},
            colorlinks=true,
            citecolor=blue,
            urlcolor=blue,
            linkcolor=magenta,
            pdfborder={0 0 0}}
\urlstyle{same}  % don't use monospace font for urls
\setlength{\parindent}{0pt}
\setlength{\parskip}{6pt plus 2pt minus 1pt}
\setlength{\emergencystretch}{3em}  % prevent overfull lines
\setcounter{secnumdepth}{0}

\title{The Lexer}
\date{}

\begin{document}
\maketitle

psilo has a comparatively simple grammar and lexing it is fairly
straightforward. Parsec has a number of utilities for accomplishing
exactly what I wish to accomplish so I happily and humbly defer to its
facilities.

The main purpose of this module is to remove some clutter from
\texttt{Parser}.

\begin{Shaded}
\begin{Highlighting}[]
\KeywordTok{module} \DataTypeTok{Lexer} \KeywordTok{where}

\KeywordTok{import }\DataTypeTok{Text.Parsec}
\KeywordTok{import }\DataTypeTok{Text.Parsec.String} \NormalTok{(}\DataTypeTok{Parser}\NormalTok{)}
\KeywordTok{import }\DataTypeTok{Text.Parsec.Language} \NormalTok{(emptyDef)}
\KeywordTok{import qualified} \DataTypeTok{Text.Parsec.Token} \KeywordTok{as} \DataTypeTok{Tok}

\OtherTok{lexer ::} \DataTypeTok{Tok.TokenParser} \NormalTok{()}
\NormalTok{lexer }\FunctionTok{=} \NormalTok{Tok.makeTokenParser style}
    \KeywordTok{where} \NormalTok{ops }\FunctionTok{=} \NormalTok{[]}
          \NormalTok{names }\FunctionTok{=} \NormalTok{[}\StringTok{"\textbackslash{}\textbackslash{}"}\NormalTok{,}\StringTok{"::"}\NormalTok{,}\StringTok{"let"}\NormalTok{,}\StringTok{"apply"}\NormalTok{]}
          \NormalTok{idStarts }\FunctionTok{=} \NormalTok{letter }\FunctionTok{<|>} \NormalTok{char }\CharTok{'_'}
          \NormalTok{idLetters }\FunctionTok{=} \NormalTok{letter }\FunctionTok{<|>} \NormalTok{char }\CharTok{'_'} \FunctionTok{<|>} \NormalTok{digit }\FunctionTok{<|>} \NormalTok{char }\CharTok{'-'}
                  \FunctionTok{<|>} \NormalTok{char }\CharTok{'+'} \FunctionTok{<|>} \NormalTok{char }\CharTok{'?'} \FunctionTok{<|>} \NormalTok{char }\CharTok{':'} \FunctionTok{<|>} \NormalTok{char }\CharTok{'&'}
          \NormalTok{opStarts }\FunctionTok{=} \NormalTok{oneOf }\StringTok{"!$%&|*+-/:<=>?@^_~#"}
          \NormalTok{opLetters }\FunctionTok{=} \NormalTok{oneOf }\StringTok{"!$%&|*+-/:<=>?@^_~#"} \FunctionTok{<|>} \NormalTok{letter }\FunctionTok{<|>} \NormalTok{digit }\FunctionTok{<|>} \NormalTok{char }\CharTok{'-'} \FunctionTok{<|>} \NormalTok{char }\CharTok{'_'}
          \NormalTok{style }\FunctionTok{=} \NormalTok{emptyDef \{}
                      \NormalTok{Tok.commentLine }\FunctionTok{=} \StringTok{";"}
                    \NormalTok{, Tok.reservedOpNames }\FunctionTok{=} \NormalTok{ops}
                    \NormalTok{, Tok.reservedNames }\FunctionTok{=} \NormalTok{names}
                    \NormalTok{, Tok.caseSensitive }\FunctionTok{=} \DataTypeTok{True}
                    \NormalTok{, Tok.identStart }\FunctionTok{=} \NormalTok{idStarts}
                    \NormalTok{, Tok.identLetter }\FunctionTok{=} \NormalTok{idLetters}
                    \NormalTok{, Tok.opStart }\FunctionTok{=} \NormalTok{opStarts}
                    \NormalTok{, Tok.opLetter }\FunctionTok{=} \NormalTok{opLetters}
                    \NormalTok{, Tok.commentStart }\FunctionTok{=} \StringTok{"/*"}
                    \NormalTok{, Tok.commentEnd }\FunctionTok{=} \StringTok{"*/"}
          \NormalTok{\}}

\OtherTok{integer ::} \DataTypeTok{Parser} \DataTypeTok{Integer}
\NormalTok{integer }\FunctionTok{=} \NormalTok{Tok.integer lexer}

\OtherTok{parens ::} \DataTypeTok{Parser} \NormalTok{a }\OtherTok{->} \DataTypeTok{Parser} \NormalTok{a}
\NormalTok{parens }\FunctionTok{=} \NormalTok{Tok.parens lexer}

\OtherTok{whitespace ::} \DataTypeTok{Parser} \NormalTok{()}
\NormalTok{whitespace }\FunctionTok{=} \NormalTok{Tok.whiteSpace lexer}

\OtherTok{nl ::} \DataTypeTok{Parser} \NormalTok{()}
\NormalTok{nl }\FunctionTok{=} \NormalTok{skipMany newline}

\OtherTok{reserved ::} \DataTypeTok{String} \OtherTok{->} \DataTypeTok{Parser} \NormalTok{()}
\NormalTok{reserved }\FunctionTok{=} \NormalTok{Tok.reserved lexer}

\OtherTok{reservedOp ::} \DataTypeTok{String} \OtherTok{->} \DataTypeTok{Parser} \NormalTok{()}
\NormalTok{reservedOp }\FunctionTok{=} \NormalTok{Tok.reservedOp lexer}

\OtherTok{identifier ::} \DataTypeTok{Parser} \DataTypeTok{String}
\NormalTok{identifier }\FunctionTok{=} \NormalTok{Tok.identifier lexer}

\OtherTok{operator ::} \DataTypeTok{Parser} \DataTypeTok{String}
\NormalTok{operator }\FunctionTok{=} \NormalTok{Tok.operator lexer}
\end{Highlighting}
\end{Shaded}

\end{document}
