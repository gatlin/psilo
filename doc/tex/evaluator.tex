\documentclass[]{article}
\usepackage{lmodern}
\usepackage{amssymb,amsmath}
\usepackage{ifxetex,ifluatex}
\usepackage{fixltx2e} % provides \textsubscript
\ifnum 0\ifxetex 1\fi\ifluatex 1\fi=0 % if pdftex
  \usepackage[T1]{fontenc}
  \usepackage[utf8]{inputenc}
\else % if luatex or xelatex
  \ifxetex
    \usepackage{mathspec}
    \usepackage{xltxtra,xunicode}
  \else
    \usepackage{fontspec}
  \fi
  \defaultfontfeatures{Mapping=tex-text,Scale=MatchLowercase}
  \newcommand{\euro}{€}
\fi
% use upquote if available, for straight quotes in verbatim environments
\IfFileExists{upquote.sty}{\usepackage{upquote}}{}
% use microtype if available
\IfFileExists{microtype.sty}{%
\usepackage{microtype}
\UseMicrotypeSet[protrusion]{basicmath} % disable protrusion for tt fonts
}{}
\usepackage{color}
\usepackage{fancyvrb}
\newcommand{\VerbBar}{|}
\newcommand{\VERB}{\Verb[commandchars=\\\{\}]}
\DefineVerbatimEnvironment{Highlighting}{Verbatim}{commandchars=\\\{\}}
% Add ',fontsize=\small' for more characters per line
\newenvironment{Shaded}{}{}
\newcommand{\KeywordTok}[1]{\textcolor[rgb]{0.00,0.44,0.13}{\textbf{{#1}}}}
\newcommand{\DataTypeTok}[1]{\textcolor[rgb]{0.56,0.13,0.00}{{#1}}}
\newcommand{\DecValTok}[1]{\textcolor[rgb]{0.25,0.63,0.44}{{#1}}}
\newcommand{\BaseNTok}[1]{\textcolor[rgb]{0.25,0.63,0.44}{{#1}}}
\newcommand{\FloatTok}[1]{\textcolor[rgb]{0.25,0.63,0.44}{{#1}}}
\newcommand{\CharTok}[1]{\textcolor[rgb]{0.25,0.44,0.63}{{#1}}}
\newcommand{\StringTok}[1]{\textcolor[rgb]{0.25,0.44,0.63}{{#1}}}
\newcommand{\CommentTok}[1]{\textcolor[rgb]{0.38,0.63,0.69}{\textit{{#1}}}}
\newcommand{\OtherTok}[1]{\textcolor[rgb]{0.00,0.44,0.13}{{#1}}}
\newcommand{\AlertTok}[1]{\textcolor[rgb]{1.00,0.00,0.00}{\textbf{{#1}}}}
\newcommand{\FunctionTok}[1]{\textcolor[rgb]{0.02,0.16,0.49}{{#1}}}
\newcommand{\RegionMarkerTok}[1]{{#1}}
\newcommand{\ErrorTok}[1]{\textcolor[rgb]{1.00,0.00,0.00}{\textbf{{#1}}}}
\newcommand{\NormalTok}[1]{{#1}}
\ifxetex
  \usepackage[setpagesize=false, % page size defined by xetex
              unicode=false, % unicode breaks when used with xetex
              xetex]{hyperref}
\else
  \usepackage[unicode=true]{hyperref}
\fi
\hypersetup{breaklinks=true,
            bookmarks=true,
            pdfauthor={},
            pdftitle={The psilo Virtual Machine},
            colorlinks=true,
            citecolor=blue,
            urlcolor=blue,
            linkcolor=magenta,
            pdfborder={0 0 0}}
\urlstyle{same}  % don't use monospace font for urls
\setlength{\parindent}{0pt}
\setlength{\parskip}{6pt plus 2pt minus 1pt}
\setlength{\emergencystretch}{3em}  % prevent overfull lines
\setcounter{secnumdepth}{0}

\title{The psilo Virtual Machine}
\date{}

\begin{document}
\maketitle

The virtual machine consists of two things:

\begin{itemize}
\itemsep1pt\parskip0pt\parsep0pt
\item
  a statically scoped mapping from symbols to store locations (the
  ``environment''); and
\item
  a persistent mapping from locations to values (the ``store'').
\end{itemize}

Thus a \texttt{Machine} is a monad composed of \texttt{ReaderT} and
\texttt{StateT} monad transformers.

The Reader monad permits function-local overwriting of the contained
state which is automatically rolled back -- precisely the behavior we
want out of our lexical environment.

The State monad, on the other hand, is persistent until the end of the
machine's execution and thus handles dynamic scope and state.

\subsection{Imports and language
extensions}\label{imports-and-language-extensions}

\begin{Shaded}
\begin{Highlighting}[]
\OtherTok{\{-# LANGUAGE DeriveFunctor #-\}}
\OtherTok{\{-# LANGUAGE DeriveFoldable #-\}}
\OtherTok{\{-# LANGUAGE DeriveTraversable #-\}}
\OtherTok{\{-# LANGUAGE StandaloneDeriving #-\}}
\OtherTok{\{-# LANGUAGE TypeSynonymInstances #-\}}
\OtherTok{\{-# LANGUAGE FlexibleInstances #-\}}
\OtherTok{\{-# LANGUAGE OverlappingInstances #-\}}
\OtherTok{\{-# LANGUAGE GeneralizedNewtypeDeriving #-\}}
\OtherTok{\{-# LANGUAGE ExistentialQuantification #-\}}

\KeywordTok{module} \DataTypeTok{Evaluator} \KeywordTok{where}

\KeywordTok{import }\DataTypeTok{Control.Monad.Free}
\KeywordTok{import }\DataTypeTok{Prelude} \KeywordTok{hiding} \NormalTok{(log,lookup)}
\KeywordTok{import }\DataTypeTok{Control.Monad}
\KeywordTok{import }\DataTypeTok{Control.Monad.State}
\KeywordTok{import }\DataTypeTok{Control.Monad.Free}
\KeywordTok{import }\DataTypeTok{Control.Monad.Trans}
\KeywordTok{import }\DataTypeTok{Control.Monad.Reader}
\KeywordTok{import qualified} \DataTypeTok{Data.Map.Strict} \KeywordTok{as} \DataTypeTok{Map}
\KeywordTok{import qualified} \DataTypeTok{Data.IntMap.Strict} \KeywordTok{as} \DataTypeTok{IntMap}
\KeywordTok{import }\DataTypeTok{Data.Foldable} \NormalTok{(}\DataTypeTok{Foldable}\NormalTok{, fold)}
\KeywordTok{import }\DataTypeTok{Data.Traversable} \NormalTok{(}\DataTypeTok{Traversable}\NormalTok{, sequence)}
\KeywordTok{import }\DataTypeTok{Data.List} \NormalTok{(intersperse)}

\KeywordTok{import }\DataTypeTok{Parser}
\KeywordTok{import }\DataTypeTok{Syntax}
\end{Highlighting}
\end{Shaded}

\subsection{The Machine}\label{the-machine}

Borrowing (stealing?) from Krishnamurthi's inimitable
\href{http://cs.brown.edu/~sk/Publications/Books/ProgLangs/}{Programming
Languages: Application and Interpretation} the environment does not map
symbols to values but to \emph{locations} in the store. The store, then,
maps location to values.

\begin{Shaded}
\begin{Highlighting}[]
\KeywordTok{type} \DataTypeTok{Location} \FunctionTok{=} \DataTypeTok{Int}
\KeywordTok{data} \DataTypeTok{Value} \FunctionTok{=} \NormalTok{forall a }\FunctionTok{.} \DataTypeTok{Show} \NormalTok{a }\OtherTok{=>} \DataTypeTok{VClos} \NormalTok{\{}\OtherTok{ vSym  ::} \NormalTok{[}\DataTypeTok{Symbol}\NormalTok{]}
                                        \NormalTok{,}\OtherTok{ vBody ::} \NormalTok{(}\DataTypeTok{Expr} \NormalTok{a)}
                                        \NormalTok{,}\OtherTok{ vEnv  ::} \DataTypeTok{Environment}
                                        \NormalTok{\}}
           \FunctionTok{|} \DataTypeTok{VSym} \DataTypeTok{Symbol}
           \FunctionTok{|} \DataTypeTok{VNum}  \NormalTok{\{}\OtherTok{ unNum ::} \DataTypeTok{Integer} \NormalTok{\}}
           \FunctionTok{|} \DataTypeTok{VBool} \NormalTok{\{}\OtherTok{ unBool ::} \DataTypeTok{Bool} \NormalTok{\}}
           \FunctionTok{|} \DataTypeTok{VList} \NormalTok{[}\DataTypeTok{Value}\NormalTok{]}
           \FunctionTok{|} \DataTypeTok{VNil}

\KeywordTok{instance} \DataTypeTok{Show} \DataTypeTok{Value} \KeywordTok{where}
    \NormalTok{show (}\DataTypeTok{VSym} \NormalTok{s)   }\FunctionTok{=} \StringTok{"'"} \FunctionTok{++} \NormalTok{s}
    \NormalTok{show (}\DataTypeTok{VNum} \NormalTok{n)   }\FunctionTok{=} \NormalTok{show n}
    \NormalTok{show (}\DataTypeTok{VBool} \NormalTok{b)  }\FunctionTok{=} \KeywordTok{if} \NormalTok{b }\KeywordTok{then} \StringTok{"#t"} \KeywordTok{else} \StringTok{"#f"}
    \NormalTok{show (}\DataTypeTok{VNil}\NormalTok{)     }\FunctionTok{=} \StringTok{"(nil)"}
    \NormalTok{show (}\DataTypeTok{VClos} \NormalTok{_ _ e)  }\FunctionTok{=} \StringTok{"<function> with Environment: "} \FunctionTok{++} \NormalTok{(show e)}
    \NormalTok{show (}\DataTypeTok{VList} \NormalTok{xs) }\FunctionTok{=} \NormalTok{concat }\FunctionTok{$} \NormalTok{map show xs}

\KeywordTok{type} \DataTypeTok{Environment} \FunctionTok{=} \DataTypeTok{Map.Map} \DataTypeTok{Symbol} \DataTypeTok{Int}
\KeywordTok{type} \DataTypeTok{Store}       \FunctionTok{=} \DataTypeTok{IntMap.IntMap} \DataTypeTok{Value}

\NormalTok{emptyEnv }\FunctionTok{=} \NormalTok{Map.empty}
\NormalTok{emptyStore }\FunctionTok{=} \NormalTok{IntMap.empty}
\end{Highlighting}
\end{Shaded}

The store must also keep track of how many locations it has handed out.
As the \texttt{StateT} monad can only hold one value as state, I wrap a
\texttt{Store} and an \texttt{Int} together in one data type.

\begin{Shaded}
\begin{Highlighting}[]
\KeywordTok{data} \DataTypeTok{MStore} \FunctionTok{=} \DataTypeTok{MStore} \NormalTok{\{}\OtherTok{ mStore ::} \DataTypeTok{Store}
                     \NormalTok{,}\OtherTok{ mLoc   ::} \DataTypeTok{Int}
                     \NormalTok{\}}
    \KeywordTok{deriving} \DataTypeTok{Show}
\end{Highlighting}
\end{Shaded}

I do the same thing with \texttt{Environment} defensively in case I need
to store more data in the \texttt{ReaderT} in the future.

\begin{Shaded}
\begin{Highlighting}[]
\KeywordTok{data} \DataTypeTok{MEnv} \FunctionTok{=} \DataTypeTok{MEnv} \NormalTok{\{}\OtherTok{ mEnv ::} \DataTypeTok{Environment} \NormalTok{\}}
    \KeywordTok{deriving} \DataTypeTok{Show}
\end{Highlighting}
\end{Shaded}

Behold: the \texttt{Machine} monad, a stack of monad transformers.

\begin{Shaded}
\begin{Highlighting}[]
\KeywordTok{newtype} \DataTypeTok{Machine} \NormalTok{a }\FunctionTok{=} \DataTypeTok{M} \NormalTok{\{}\OtherTok{ runM ::} \DataTypeTok{ReaderT} \DataTypeTok{MEnv} \NormalTok{(}\DataTypeTok{StateT} \DataTypeTok{MStore} \DataTypeTok{IO}\NormalTok{) a \}}
    \KeywordTok{deriving} \NormalTok{(}\DataTypeTok{Monad}\NormalTok{, }\DataTypeTok{MonadIO}\NormalTok{, }\DataTypeTok{MonadState} \DataTypeTok{MStore}\NormalTok{, }\DataTypeTok{MonadReader} \DataTypeTok{MEnv}\NormalTok{)}
\end{Highlighting}
\end{Shaded}

\begin{Shaded}
\begin{Highlighting}[]
\OtherTok{initialStore ::} \DataTypeTok{MStore}
\NormalTok{initialStore }\FunctionTok{=} \DataTypeTok{MStore} \NormalTok{\{ mStore }\FunctionTok{=} \NormalTok{emptyStore}
                      \NormalTok{, mLoc   }\FunctionTok{=} \DecValTok{1}
                      \NormalTok{\}}
\end{Highlighting}
\end{Shaded}

\begin{Shaded}
\begin{Highlighting}[]
\OtherTok{initialEnv ::} \DataTypeTok{MEnv}
\NormalTok{initialEnv }\FunctionTok{=} \DataTypeTok{MEnv} \NormalTok{\{ mEnv }\FunctionTok{=} \NormalTok{emptyEnv \}}
\end{Highlighting}
\end{Shaded}

With the above default initial states for the environment and the store,
I'm ready to define the mapping from a \texttt{Machine} to \texttt{IO},
which is essentially just calling the various monad transformer
\texttt{run} functions in succession.

\begin{Shaded}
\begin{Highlighting}[]
\OtherTok{runMachineWithStore ::} \DataTypeTok{Machine} \NormalTok{a }\OtherTok{->} \DataTypeTok{MStore} \OtherTok{->} \DataTypeTok{IO} \NormalTok{(a, }\DataTypeTok{MStore}\NormalTok{)}
\NormalTok{runMachineWithStore k st }\FunctionTok{=} \NormalTok{runStateT (runReaderT (runM k) initialEnv) st}
\end{Highlighting}
\end{Shaded}

\begin{Shaded}
\begin{Highlighting}[]
\OtherTok{runMachine ::} \DataTypeTok{Machine} \NormalTok{a }\OtherTok{->} \DataTypeTok{IO} \NormalTok{(a, }\DataTypeTok{MStore}\NormalTok{)}
\NormalTok{runMachine k }\FunctionTok{=} \NormalTok{runMachineWithStore k initialStore}
\end{Highlighting}
\end{Shaded}

Now all that is left is a means of building a \texttt{Machine} from
psilo code.

\subsection{The operation language}\label{the-operation-language}

Executing a psilo program on this machine amounts to:

\begin{enumerate}
\def\labelenumi{\arabic{enumi}.}
\itemsep1pt\parskip0pt\parsep0pt
\item
  Unwinding \texttt{Expr} values and concurrently
\item
  Building the corresponding \texttt{Machine} values.
\end{enumerate}

To aid in this second step I define an intermediate operation language,
\texttt{Op}, which will encapsulate some common machine-oriented tasks.
This should simplify the proper interpreter function.

\texttt{Op} allows for manipulation of the environment and store, and
also provides fresh store locations on demand.

\begin{Shaded}
\begin{Highlighting}[]
\KeywordTok{data} \DataTypeTok{OpF} \NormalTok{k}
    \FunctionTok{=} \DataTypeTok{Bind}   \DataTypeTok{Symbol} \DataTypeTok{Location} \NormalTok{k}
    \FunctionTok{|} \DataTypeTok{Lookup} \DataTypeTok{Symbol} \NormalTok{(}\DataTypeTok{Location} \OtherTok{->} \NormalTok{k)}
    \FunctionTok{|} \DataTypeTok{Store}  \DataTypeTok{Location} \DataTypeTok{Value} \NormalTok{k}
    \FunctionTok{|} \DataTypeTok{Fetch}  \DataTypeTok{Location} \NormalTok{(}\DataTypeTok{Value} \OtherTok{->} \NormalTok{k)}
    \FunctionTok{|} \DataTypeTok{Delete} \DataTypeTok{Location} \NormalTok{k}
    \FunctionTok{|} \DataTypeTok{Fresh}  \NormalTok{(}\DataTypeTok{Location} \OtherTok{->} \NormalTok{k)}
    \KeywordTok{deriving} \DataTypeTok{Functor}
\end{Highlighting}
\end{Shaded}

\begin{Shaded}
\begin{Highlighting}[]
\KeywordTok{type} \DataTypeTok{Op} \FunctionTok{=} \DataTypeTok{Free} \DataTypeTok{OpF}
\end{Highlighting}
\end{Shaded}

Nevermind the \texttt{Free} constructor for now.

The utility of the following convenience functions will be clearer when
we get to the interpreter. A crude explanation is that these are the
``commands'' we will write \texttt{Op} programs in.

\begin{Shaded}
\begin{Highlighting}[]
\OtherTok{bind ::} \DataTypeTok{Symbol} \OtherTok{->} \DataTypeTok{Location} \OtherTok{->} \DataTypeTok{Op} \NormalTok{()}
\NormalTok{bind s l }\FunctionTok{=} \NormalTok{liftF }\FunctionTok{$} \DataTypeTok{Bind} \NormalTok{s l ()}
\end{Highlighting}
\end{Shaded}

\begin{Shaded}
\begin{Highlighting}[]
\NormalTok{lookup}\OtherTok{ ::} \DataTypeTok{Symbol} \OtherTok{->} \DataTypeTok{Op} \DataTypeTok{Location}
\NormalTok{lookup s }\FunctionTok{=} \NormalTok{liftF }\FunctionTok{$} \DataTypeTok{Lookup} \NormalTok{s id}
\end{Highlighting}
\end{Shaded}

\begin{Shaded}
\begin{Highlighting}[]
\OtherTok{store ::} \DataTypeTok{Location} \OtherTok{->} \DataTypeTok{Value} \OtherTok{->} \DataTypeTok{Op} \NormalTok{()}
\NormalTok{store l v }\FunctionTok{=} \NormalTok{liftF }\FunctionTok{$} \DataTypeTok{Store} \NormalTok{l v ()}
\end{Highlighting}
\end{Shaded}

\begin{Shaded}
\begin{Highlighting}[]
\OtherTok{fetch ::} \DataTypeTok{Location} \OtherTok{->} \DataTypeTok{Op} \DataTypeTok{Value}
\NormalTok{fetch l }\FunctionTok{=} \NormalTok{liftF }\FunctionTok{$} \DataTypeTok{Fetch} \NormalTok{l id}
\end{Highlighting}
\end{Shaded}

\begin{Shaded}
\begin{Highlighting}[]
\OtherTok{delete ::} \DataTypeTok{Location} \OtherTok{->} \DataTypeTok{Op} \NormalTok{()}
\NormalTok{delete l }\FunctionTok{=} \NormalTok{liftF }\FunctionTok{$} \DataTypeTok{Delete} \NormalTok{l ()}
\end{Highlighting}
\end{Shaded}

\begin{Shaded}
\begin{Highlighting}[]
\OtherTok{fresh ::} \DataTypeTok{Op} \DataTypeTok{Location}
\NormalTok{fresh }\FunctionTok{=} \NormalTok{liftF }\FunctionTok{$} \DataTypeTok{Fresh} \NormalTok{id}
\end{Highlighting}
\end{Shaded}

\begin{Shaded}
\begin{Highlighting}[]
\OtherTok{lookupVar ::} \DataTypeTok{Symbol} \OtherTok{->} \DataTypeTok{MEnv} \OtherTok{->} \DataTypeTok{Location}
\NormalTok{lookupVar sym (}\DataTypeTok{MEnv} \NormalTok{env) }\FunctionTok{=} \NormalTok{env }\FunctionTok{Map.!} \NormalTok{sym}
\end{Highlighting}
\end{Shaded}

\begin{Shaded}
\begin{Highlighting}[]
\OtherTok{bindVar ::} \DataTypeTok{Symbol} \OtherTok{->} \DataTypeTok{Location} \OtherTok{->} \DataTypeTok{MEnv} \OtherTok{->} \DataTypeTok{MEnv}
\NormalTok{bindVar sym loc (}\DataTypeTok{MEnv} \NormalTok{env) }\FunctionTok{=} \DataTypeTok{MEnv} \FunctionTok{$} \NormalTok{Map.insert sym loc env}
\end{Highlighting}
\end{Shaded}

\subsection{The \texttt{Op} interpreter and the \texttt{Free}
monad}\label{the-op-interpreter-and-the-free-monad}

\begin{Shaded}
\begin{Highlighting}[]
\OtherTok{runOp ::} \DataTypeTok{Op} \NormalTok{a }\OtherTok{->} \DataTypeTok{Machine} \NormalTok{a}
\end{Highlighting}
\end{Shaded}

For each branch of our \texttt{OpF} data type definition, we have a case
for the \texttt{Op} interpreter to handle.

\texttt{Op} and \texttt{OpF} are slightly different: the former is the
latter transformed by the \texttt{Free} monad type constructor.
\texttt{Free} is exactly that: you give it a functor value and get a
monad for ``free''; ie, you get a data type which implements
\texttt{\textgreater{}\textgreater{}=} and \texttt{return}.

However, \texttt{Free} monads all get the same generic implementations
of \texttt{\textgreater{}\textgreater{}=} and \texttt{bind}. All the
semantics of your data type must be specified in a \emph{run} function
of some kind which breaks down these aggregate values to build some
result. \texttt{runOp} is such a function for \texttt{Op}.

\texttt{Free} creates types which have a base case for storing ``Pure''
values (in Haskell, the \texttt{return} function may also be called
\texttt{pure} because it wraps a \emph{pure} value in a monadic
context). \texttt{Free} values all have a continuation argument (which I
call \texttt{next}, conventionally), whereas \texttt{Pure} values do
not: they are leaves of a syntax tree.

\begin{Shaded}
\begin{Highlighting}[]
\NormalTok{runOp (}\DataTypeTok{Pure} \NormalTok{v) }\FunctionTok{=} \NormalTok{return v}
\end{Highlighting}
\end{Shaded}

All the other value constructors are wrapped in \texttt{Free}.

\texttt{bind}ing is the act of associating a symbol with a location in
memory.

\begin{Shaded}
\begin{Highlighting}[]
\NormalTok{runOp (}\DataTypeTok{Free} \NormalTok{(}\DataTypeTok{Bind} \NormalTok{sym loc next)) }\FunctionTok{=} \KeywordTok{do}
    \NormalTok{env }\OtherTok{<-} \NormalTok{ask}
    \NormalTok{local (bindVar sym loc) }\FunctionTok{$} \NormalTok{runOp next}
\end{Highlighting}
\end{Shaded}

The inverse operation is called a \texttt{lookup} in our parlance:

\begin{Shaded}
\begin{Highlighting}[]
\NormalTok{runOp (}\DataTypeTok{Free} \NormalTok{(}\DataTypeTok{Lookup} \NormalTok{sym next)) }\FunctionTok{=} \KeywordTok{do}
    \NormalTok{loc }\OtherTok{<-} \NormalTok{asks (lookupVar sym)}
    \NormalTok{runOp }\FunctionTok{$} \NormalTok{next loc}
\end{Highlighting}
\end{Shaded}

If we have a location (perhaps by calling \texttt{fresh}) to store a
value in - and a value - we can associate them by \texttt{get}ting the
state, picking out the \texttt{Store}, and inserting our location and
value in the map. Then we \texttt{put} the state back in and continue on
our merry way.

\begin{Shaded}
\begin{Highlighting}[]
\NormalTok{runOp (}\DataTypeTok{Free} \NormalTok{(}\DataTypeTok{Store} \NormalTok{loc val next)) }\FunctionTok{=} \KeywordTok{do}
    \NormalTok{state  }\OtherTok{<-} \NormalTok{get}
    \NormalTok{sto    }\OtherTok{<-} \NormalTok{return }\FunctionTok{$} \NormalTok{mStore state}
    \NormalTok{sto'   }\OtherTok{<-} \NormalTok{return }\FunctionTok{$} \NormalTok{IntMap.insert loc val sto}
    \NormalTok{put }\FunctionTok{$} \NormalTok{state \{ mStore }\FunctionTok{=} \NormalTok{sto' \}}
    \NormalTok{runOp next}
\end{Highlighting}
\end{Shaded}

By this point should be able to figure out what \texttt{fetch} does.

\begin{Shaded}
\begin{Highlighting}[]
\NormalTok{runOp (}\DataTypeTok{Free} \NormalTok{(}\DataTypeTok{Fetch} \NormalTok{loc next)) }\FunctionTok{=} \KeywordTok{do}
    \NormalTok{state  }\OtherTok{<-} \NormalTok{get}
    \NormalTok{sto    }\OtherTok{<-} \NormalTok{return }\FunctionTok{$} \NormalTok{mStore state}
    \NormalTok{val    }\OtherTok{<-} \NormalTok{return }\FunctionTok{$} \NormalTok{sto }\FunctionTok{IntMap.!} \NormalTok{loc}
    \NormalTok{runOp }\FunctionTok{$} \NormalTok{next val}
\end{Highlighting}
\end{Shaded}

\begin{Shaded}
\begin{Highlighting}[]
\NormalTok{runOp (}\DataTypeTok{Free} \NormalTok{(}\DataTypeTok{Delete} \NormalTok{loc next)) }\FunctionTok{=} \KeywordTok{do}
    \NormalTok{state }\OtherTok{<-} \NormalTok{get}
    \NormalTok{sto   }\OtherTok{<-} \NormalTok{return }\FunctionTok{$} \NormalTok{mStore state}
    \NormalTok{sto'  }\OtherTok{<-} \NormalTok{return }\FunctionTok{$} \NormalTok{IntMap.delete loc sto}
    \NormalTok{put }\FunctionTok{$} \NormalTok{state \{ mStore }\FunctionTok{=} \NormalTok{sto' \}}
    \NormalTok{runOp next}
\end{Highlighting}
\end{Shaded}

\texttt{fresh} gets the state, extracts the location, increments it,
puts it back in, and returns the original.

\begin{Shaded}
\begin{Highlighting}[]
\NormalTok{runOp (}\DataTypeTok{Free} \NormalTok{(}\DataTypeTok{Fresh} \NormalTok{next)) }\FunctionTok{=} \KeywordTok{do}
    \NormalTok{state }\OtherTok{<-} \NormalTok{get}
    \NormalTok{loc   }\OtherTok{<-} \NormalTok{return }\FunctionTok{$} \NormalTok{mLoc state}
    \NormalTok{put }\FunctionTok{$} \NormalTok{state \{ mLoc }\FunctionTok{=} \NormalTok{(loc }\FunctionTok{+} \DecValTok{1}\NormalTok{) \}}
    \NormalTok{runOp }\FunctionTok{$} \NormalTok{next loc}
\end{Highlighting}
\end{Shaded}

To illustrate what \texttt{Op} code looks like have a look at
\texttt{opTest}:

\begin{Shaded}
\begin{Highlighting}[]
\OtherTok{opTest ::} \DataTypeTok{Machine} \NormalTok{()}
\NormalTok{opTest }\FunctionTok{=} \NormalTok{runOp }\FunctionTok{$} \KeywordTok{do}
    \NormalTok{loc1 }\OtherTok{<-} \NormalTok{fresh}
    \NormalTok{bind }\StringTok{"huh"} \NormalTok{loc1}
    \NormalTok{store loc1 }\FunctionTok{$} \DataTypeTok{VNum} \DecValTok{5}
    \NormalTok{(}\DataTypeTok{VNum} \NormalTok{val) }\OtherTok{<-} \NormalTok{lookup }\StringTok{"huh"} \FunctionTok{>>=} \NormalTok{fetch}
    \NormalTok{bind }\StringTok{"huh"} \DecValTok{0}
    \NormalTok{store }\DecValTok{0} \FunctionTok{$} \DataTypeTok{VNum} \NormalTok{(val }\FunctionTok{*} \DecValTok{2}\NormalTok{)}
    \NormalTok{return ()}
\end{Highlighting}
\end{Shaded}

\subsection{Interpreting psilo}\label{interpreting-psilo}

Now that we have a static environment and a dynamic store, a machine
which holds them, and a low-level operation language to control the
machine, we can now set ourselves to interpreting psilo.

As stated elsewhere, executing psilo programs is the act of

\begin{enumerate}
\def\labelenumi{\arabic{enumi}.}
\itemsep1pt\parskip0pt\parsep0pt
\item
  transforming \texttt{Expr} values to \texttt{Machine} values and
\item
  unwinding \texttt{Machine} values.
\end{enumerate}

The function \texttt{interpret} handles the first part. You can even
tell by its type: \texttt{Expr a -\textgreater{} Machine Value}.

\begin{Shaded}
\begin{Highlighting}[]
\OtherTok{interpret ::} \DataTypeTok{Show} \NormalTok{a }\OtherTok{=>} \DataTypeTok{Expr} \NormalTok{a }\OtherTok{->} \DataTypeTok{Machine} \DataTypeTok{Value}
\end{Highlighting}
\end{Shaded}

\texttt{Expr} is defined as \texttt{type Expr = Free AST}. Since
\texttt{Expr} is a \texttt{Free} monad, as with \texttt{Op}, we handle
the base case of being handed a \texttt{Pure} value. In this case, we
return \texttt{NilV}.

\begin{Shaded}
\begin{Highlighting}[]
\NormalTok{interpret (}\DataTypeTok{Pure} \NormalTok{v) }\FunctionTok{=} \NormalTok{return }\DataTypeTok{VNil}
\end{Highlighting}
\end{Shaded}

The rest of the interpreter is remarkably simple. Each case corresponds
to a branch in our \texttt{AST} definition.

\begin{Shaded}
\begin{Highlighting}[]
\NormalTok{interpret (}\DataTypeTok{Free} \NormalTok{(}\DataTypeTok{ABoolean} \NormalTok{b)) }\FunctionTok{=} \NormalTok{return }\FunctionTok{$} \DataTypeTok{VBool} \NormalTok{b}
\end{Highlighting}
\end{Shaded}

\begin{Shaded}
\begin{Highlighting}[]
\NormalTok{interpret (}\DataTypeTok{Free} \NormalTok{(}\DataTypeTok{AInteger} \NormalTok{n)) }\FunctionTok{=} \NormalTok{return }\FunctionTok{$} \DataTypeTok{VNum} \NormalTok{n}
\end{Highlighting}
\end{Shaded}

Symbols may have prefixes or suffixes (well, eventually) which modify
the semantic value of the symbol but not the actual raw value. For
example, a \texttt{:\&} suffix tells the compiler that the symbol is a
shared reference and may be safely ignored.

While this will be more nuanced or sophisticated in the future, in this
evaluator at least we may safely ignore all suffixes.

\begin{Shaded}
\begin{Highlighting}[]
\NormalTok{interpret (}\DataTypeTok{Free} \NormalTok{(}\DataTypeTok{ASymbol}  \NormalTok{s)) }\FunctionTok{=} \KeywordTok{do}
    \NormalTok{val }\OtherTok{<-} \NormalTok{runOp }\FunctionTok{$} \NormalTok{lookup s }\FunctionTok{>>=} \NormalTok{fetch}
    \NormalTok{return val}
\end{Highlighting}
\end{Shaded}

\begin{Shaded}
\begin{Highlighting}[]
\NormalTok{interpret (}\DataTypeTok{Free} \NormalTok{(}\DataTypeTok{AList} \NormalTok{xs)) }\FunctionTok{=} \KeywordTok{do}
    \NormalTok{vals }\OtherTok{<-} \NormalTok{forM xs }\FunctionTok{$} \NormalTok{\textbackslash{}x }\OtherTok{->} \KeywordTok{do}
        \NormalTok{x' }\OtherTok{<-} \NormalTok{return x}
        \KeywordTok{let} \NormalTok{v }\FunctionTok{=} \NormalTok{interpret x'}
        \NormalTok{v}
    \NormalTok{return }\FunctionTok{$} \DataTypeTok{VList} \NormalTok{vals}
\end{Highlighting}
\end{Shaded}

The below code for lambdas, while technically correct, has a huge
problem: it copies its \emph{entire} environment. A much smarter trick
would be to only copy that which is actually used.

Also, function application is currently very stupid. It is intended that
functions will take one argument, a list containing the actual values to
be processed. The process is simple:

\begin{enumerate}
\def\labelenumi{\arabic{enumi}.}
\item
  If the operand list is sufficiently long, zip it with the list of
  symbols in the function.
\item
  Create an \texttt{Environment} out of this zipped list.
\item
  Form a union between this new environment and the current, favoring
  the new one.
\end{enumerate}

\begin{Shaded}
\begin{Highlighting}[]
\NormalTok{interpret (}\DataTypeTok{Free} \NormalTok{(}\DataTypeTok{ALambda} \NormalTok{args body)) }\FunctionTok{=} \KeywordTok{do}
    \NormalTok{(}\DataTypeTok{MEnv} \NormalTok{currentEnv) }\OtherTok{<-} \NormalTok{ask}
    \NormalTok{return }\FunctionTok{$} \DataTypeTok{VClos} \NormalTok{args body currentEnv}
\end{Highlighting}
\end{Shaded}

During application, if we are given a symbol for an operator, check to
see if it is a built-in operator and, if applicable, simply return the
resulting \texttt{Value}.

\begin{Shaded}
\begin{Highlighting}[]
\NormalTok{interpret (}\DataTypeTok{Free} \NormalTok{(}\DataTypeTok{AApply} \NormalTok{fun args)) }\FunctionTok{=} \KeywordTok{do}
    \NormalTok{(}\DataTypeTok{VList} \NormalTok{argVals)       }\OtherTok{<-} \NormalTok{interpret args}
    \KeywordTok{case} \NormalTok{builtin fun argVals }\KeywordTok{of}
        \DataTypeTok{Just} \NormalTok{mv   }\OtherTok{->} \NormalTok{return mv}
        \DataTypeTok{Nothing}   \OtherTok{->} \KeywordTok{do}
            \NormalTok{(}\DataTypeTok{VClos} \NormalTok{syms body env) }\OtherTok{<-} \NormalTok{interpret fun}
            \NormalTok{locations }\OtherTok{<-} \NormalTok{forM argVals }\FunctionTok{$} \NormalTok{\textbackslash{}av }\OtherTok{->} \KeywordTok{do}
                \NormalTok{newLoc }\OtherTok{<-} \NormalTok{runOp }\FunctionTok{$} \NormalTok{fresh}
                \NormalTok{runOp }\FunctionTok{$} \NormalTok{store newLoc av}
                \NormalTok{return newLoc}
            \KeywordTok{let} \NormalTok{env' }\FunctionTok{=} \NormalTok{Map.fromList }\FunctionTok{$} \NormalTok{zip syms locations}
            \NormalTok{newFrame }\OtherTok{<-} \NormalTok{return }\FunctionTok{$} \NormalTok{Map.union env' env}
            \NormalTok{oldEnv   }\OtherTok{<-} \NormalTok{ask}
            \NormalTok{retVal   }\OtherTok{<-} \NormalTok{local (\textbackslash{}(}\DataTypeTok{MEnv} \NormalTok{e) }\OtherTok{->} \DataTypeTok{MEnv} \NormalTok{(Map.union newFrame e)) }\FunctionTok{$}
                            \NormalTok{interpret body}
            \CommentTok{-- clean up the store}
            \NormalTok{runOp }\FunctionTok{$} \NormalTok{forM_ locations }\FunctionTok{$} \NormalTok{\textbackslash{}loc }\OtherTok{->} \NormalTok{delete loc}
            \NormalTok{return retVal}
\end{Highlighting}
\end{Shaded}

This interpreter is flexible and powerful because we built up the
appropriate abstractions.

\subsection{Built-in operators}\label{built-in-operators}

As a final note, some symbols denote built-in operators (mostly
involving arithmetic). The following function attempts to evaluate a
built-in, returning (maybe) a \texttt{Machine Value}.

\begin{Shaded}
\begin{Highlighting}[]
\OtherTok{builtin ::} \DataTypeTok{Expr} \NormalTok{a }\OtherTok{->} \NormalTok{[}\DataTypeTok{Value}\NormalTok{] }\OtherTok{->} \DataTypeTok{Maybe} \DataTypeTok{Value}
\NormalTok{builtin (}\DataTypeTok{Free} \NormalTok{(}\DataTypeTok{ASymbol} \NormalTok{sym)) args}
    \FunctionTok{|} \NormalTok{sym }\FunctionTok{==} \StringTok{"+"}    \FunctionTok{=} \NormalTok{numOp sum args}
    \FunctionTok{|} \NormalTok{sym }\FunctionTok{==} \StringTok{"*"}    \FunctionTok{=} \NormalTok{numOp product args}
    \FunctionTok{|} \NormalTok{sym }\FunctionTok{==} \StringTok{"-"}    \FunctionTok{=} \NormalTok{numBinOp ((}\FunctionTok{-}\NormalTok{)) args}
    \FunctionTok{|} \NormalTok{sym }\FunctionTok{==} \StringTok{"/"}    \FunctionTok{=} \NormalTok{numBinOp div   args}
    \FunctionTok{|} \NormalTok{sym }\FunctionTok{==} \StringTok{"and"}  \FunctionTok{=} \DataTypeTok{Just} \FunctionTok{.} \DataTypeTok{VBool} \FunctionTok{$} \NormalTok{and (map unBool args)}
    \FunctionTok{|} \NormalTok{sym }\FunctionTok{==} \StringTok{"or"}   \FunctionTok{=} \DataTypeTok{Just} \FunctionTok{.} \DataTypeTok{VBool} \FunctionTok{$} \NormalTok{or  (map unBool args)}
    \FunctionTok{|} \NormalTok{sym }\FunctionTok{==} \StringTok{"not"}  \FunctionTok{=} \DataTypeTok{Just} \FunctionTok{.} \DataTypeTok{VBool} \FunctionTok{$} \NormalTok{not (unBool }\FunctionTok{.} \NormalTok{head }\FunctionTok{$} \NormalTok{args)}
    \FunctionTok{|} \NormalTok{otherwise     }\FunctionTok{=} \DataTypeTok{Nothing}
    \KeywordTok{where} \NormalTok{numBinOp op xs }\FunctionTok{=} \KeywordTok{let} \NormalTok{(}\DataTypeTok{VNum} \NormalTok{l) }\FunctionTok{=} \NormalTok{xs }\FunctionTok{!!} \DecValTok{0}
                               \NormalTok{(}\DataTypeTok{VNum} \NormalTok{r) }\FunctionTok{=} \NormalTok{xs }\FunctionTok{!!} \DecValTok{1}
                            \KeywordTok{in}  \DataTypeTok{Just} \FunctionTok{.} \DataTypeTok{VNum} \FunctionTok{$} \NormalTok{op l r}
          \NormalTok{numOp    op xs }\FunctionTok{=} \DataTypeTok{Just} \FunctionTok{.} \DataTypeTok{VNum} \FunctionTok{$} \NormalTok{sum (map unNum args)}
\end{Highlighting}
\end{Shaded}

\begin{Shaded}
\begin{Highlighting}[]
\NormalTok{builtin _ _   }\FunctionTok{=} \DataTypeTok{Nothing}
\end{Highlighting}
\end{Shaded}

\end{document}
